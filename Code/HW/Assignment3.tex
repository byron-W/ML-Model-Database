\documentclass{article}

\title{CDS 302 Assignment 3}
\author{Byron Washington}
\date{February 5, 2025}

\begin{document}
\maketitle
A foreign key constraint ensures that the foreign key in one relation, matches its primary key counterpart. A referential integrity constraint ensures that there is consistency between tables when manipulating data in relations. The difference between them is that a foreign key constraint is a referential integrity constraint but not all referential integrity constraints are foreign key constraints. For example, if you have two relations R1(A,B,C) and R2(C,Y,Z) where C is the same in both. The rule that would keep them consistent would be referential integrity but not a foreign key constraint because we haven't defined any keys yet.\cite{Gigoyan}

\begin{thebibliography}{1}
    \bibitem{Gigoyan} Gigoyan, Sergey. “SQL Server Referential Integrity Without Foreign Keys.” MSSQLTips.Com, 31 Mar. 2016, www.mssqltips.com/sqlservertip/4242/sql-server-referential-integrity-without-foreign-keys/. 
\end{thebibliography}
\end{document}